\documentclass[12pt, a4paper, twoside, titlepage]{article}
\usepackage{tikz}
% font size could be 10pt (default), 11pt or 12 pt
% paper size coulde be letterpaper (default), legalpaper, executivepaper,
% a4paper, a5paper or b5paper
% side coulde be oneside (default) or twoside
% columns coulde be onecolumn (default) or twocolumn
% graphics coulde be final (default) or draft
%
% titlepage coulde be notitlepage (default) or titlepage which
% makes an extra page for title
%
% paper alignment coulde be portrait (default) or landscape
%
% equations coulde be
%   default number of the equation on the rigth and equation centered
%   leqno number on the left and equation centered
%   fleqn number on the rigth and  equation on the left side
%
\usepackage{geometry}
 \geometry{
 a4paper,
 total={170mm,257mm},
 left=20mm,
 top=20mm,
 }

\title{A not so small \LaTeX{} Article Template\thanks{To your mother}}
\author{Your Name  \\
	Your Company / University  \\
	\and
	The Other Dude \\
	His Company / University \\
	}

\date{\today}
% \date{\today} date coulde be today
% \date{25.12.00} or be a certain date
% \date{ } or there is no date
\begin{document}
% Hint: \title{what ever}, \author{who care} and \date{when ever} could stand
% before or after the \begin{document} command
% BUT the \maketitle command MUST come AFTER the \begin{document} command!
%\maketitle


%\begin{abstract}
%Short introduction to subject of the paper \ldots
%\end{abstract}

%\tableofcontents % create a table of contens

\section*{WCC - R class summer 2019 - Questionnaire}

We thank Andi Lemire and Anton Schulman for their time and effort to teach R to the Women's Coding Circle. Big thanks goes to Sarada Viswanathan as well, for organizing this class!

To better understand, if the class fitted your needs and how we can improve in the future, we prepared a small questionnaire.

\begin{enumerate}
	\item What is your level of programming experience?
        \begin{figure}[!ht]
             \begin {center}
               \begin{tikzpicture}
                 %\filldraw[draw=black,color=blue] (0,0) rectangle (10,5);
                 \draw (0,0) rectangle (15,3);
               \end{tikzpicture}
             \end{center}
        \end{figure}
	\item Did you feel you are lacking math skills to follow the class?
        \begin{figure}[!ht]
             \begin {center}
               \begin{tikzpicture}
                 %\filldraw[draw=black,color=blue] (0,0) rectangle (10,5);
                 \draw (0,0) rectangle (15,3);
               \end{tikzpicture}
             \end{center}
        \end{figure}
	\item Did you get stuck during the class? If, so when?
	    \begin{figure}[!ht]
             \begin {center}
               \begin{tikzpicture}
                 %\filldraw[draw=black,color=blue] (0,0) rectangle (10,5);
                 \draw (0,0) rectangle (15,3);
               \end{tikzpicture}
             \end{center}
        \end{figure}
	\item Any other comments or suggestions for us?
	    \begin{figure}[!ht]
            \begin {center}
                \begin{tikzpicture}
                    \draw (0,0) rectangle (15,3);
                \end{tikzpicture}
            \end{center}
        \end{figure}
\end{enumerate}

\end{document}
